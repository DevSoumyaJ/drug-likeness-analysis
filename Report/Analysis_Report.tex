\documentclass[12pt]{article}
\usepackage{graphicx} % Required for inserting images
\usepackage[margin=1.2in]{geometry}
\setlength{\headheight}{0pt}
\usepackage[colorlinks=true, linkcolor=cyan, urlcolor=cyan]{hyperref}
\usepackage{booktabs}
\usepackage{float}
\usepackage{placeins}

\title{Drug-Likeness Analysis}
\author{Soumya Jyoti Sain\thanks{Code Available at \url{https://github.com/DevSoumyaJ/drug-likeness-analysis/}}}

\date{}

\begin{document}

\maketitle

\begin{abstract}
In the early stages of drug discovery, in silico evaluation of drug-likeness is a crucial step in assessing the viability of potential drug molecules. This is achieved by implementing various predefined physicochemical guidelines, such as those proposed by Lipinski, Veber, Ghose, and Egan. This project presents a comparative analysis of these guidelines by applying them to 1037 FDA-approved drugs. Molecular structures were generated from the SMILES strings, and 8 molecular descriptors were calculated using RDKit. Each filter was applied to each molecule and classified as pass or fail based on compliance with all criteria. Following this, a comparative analysis was performed, which revealed that the Lipinski and Veber filters exhibited similar permissiveness, allowing over 900 molecules to pass, whereas the Egan filter was slightly more stringent, and the Ghose filter was substantially more restrictive, with most drug molecules failing the filter. Distribution analysis showed that most drugs passed 3 filters, followed by those passing all 4. Additionally, the descriptor violations responsible for failures in the Lipinski and Ghose filters were identified and analysed. This study overall provides insights on the viability of computational filters.
\end{abstract}

\section{Introduction}
Physicochemical filters are integral to medicinal chemistry and drug discovery, with their computational nature reducing the need for resource-intensive early experimental screening. The project employs a comparative study of drug-likeness on FDA-approved drugs. By examining compliance patterns, the study highlights that these criteria serve as advisory guidelines rather than absolute requirements.

\section{Dataset}
The dataset was obtained from \href{https://pubchem.ncbi.nlm.nih.gov/}{PubChem}, containing SMILES strings and various molecular descriptors. For consistency purposes, the SMILES were first converted into RDKit-compatible molecules, and then the original molecular descriptors were replaced with RDKit molecular descriptors.


\section{Methodology}
\subsection{Descriptor Calculation}
    \begin{table}[h]
    \centering
    \caption{Molecular descriptors used in the study}
        \begin{tabular}{ll}
            \toprule
            Descriptor & Description \\
            \midrule
            Molecular Weight & Molecular mass of the compound \\
            LogP & Octanol--water partition coefficient \\
            TPSA & Topological polar surface area \\
            HBD & Number of hydrogen bond donors \\
            HBA & Number of hydrogen bond acceptors \\
            Atom Count & Total number of heavy atoms (non-hydrogen atoms) \\
            Rotatable Bonds & Number of rotatable bonds (molecular flexibility) \\
            MR & Molecular refractivity \\
            \bottomrule
        \end{tabular}
    \end{table}
\subsection{Application of Drug-Likeness Filters}
\begin{table}[h]
    \caption{Drug-likeness filters and their criteria}
    \centering
    \begin{tabular}{lll}
        \toprule
         Filter & Descriptor & Threshold \\
         \midrule
         Lipinski & Molecular Weight & $\leq 500$ Da\\
                  & LogP & $\leq 5$\\
                  & HBD & $\leq 5$\\
                  & HBA & $\leq 10$\\
        \midrule
        Ghose & Molecular Weight & 160-480 Da\\
              & LogP & -0.4-5.6\\
              & Molar Refractivity & 40-130\\
              & Atom Count & 20-70\\
        \midrule
        Veber & TPSA & $\leq 140$ \AA $^2$\\
              & Rotatable Bonds & $\leq 10$\\
        \midrule
        Egan & LogP & $\leq 5.88$ \\
             & TPSA & $\leq 131$ \AA$^2$ \\
        \bottomrule
    \end{tabular}
\end{table}

\clearpage
\subsection{Visualisation}
\begin{figure}[!htbp]
    \centering
    \label{fig:fig-1}
    \includegraphics[width=0.5\linewidth]{Visualisation/Filter-wise pass frequency comparison.png}
    \caption{Filter-wise pass frequency comparison}
\end{figure}
\begin{figure}[!htbp]
    \centering
    \label{fig:fig-2}
    \includegraphics[width=0.55\linewidth]{Visualisation/lipinski & ghose descriptor-wise fail frequency.png}
    \caption{Lipinski and Ghose descriptor-wise fail frequency}
\end{figure}
\begin{figure}[!htbp]
    \centering
    \label{fig:fig-3}
    \includegraphics[width=0.45\linewidth]{Visualisation/distribution based on filters passed.png}
    \caption{Distribution based on number of filters passed}
\end{figure}
\FloatBarrier
\pagebreak
\section{Key Observations}
\begin{enumerate}
    \item From \hyperref[fig:fig-1]{Fig.1}, we observe that Lipinski and Veber filters exhibit similar and high permissiveness, with over 900 molecules passing the filters. Egan filter imposes slightly stricter restrictions, with just under 800 molecules passing. On the contrary, the Ghose filter has markedly strict filtering conditions, excluding a significantly larger fraction of compounds, with 667 molecules failing. The rough agreement between the Lipinski, Veber and Egan filters suggest overlapping chemical spaces, aligning well with most of the approved drugs. A possible explanation for the significant rejection rate for the Ghose filter can be the extremely large or extremely small sizes of drug molecules. The specific descriptor-wise failure chart has been examined subsequently.
    \item From \hyperref[fig:fig-2]{Fig.2}, it is evident that failure under the Lipinski filter lies in the LogP constraints, followed by the Molecular Weight conditions. Implying a stringent condition on drug liphophilicity. For the Ghose filter, the dominant contributing factor responsible for the failure is the Atom Count condition, causing nearly 50\% of molecules from the dataset to fail, highlighting the stringent nature of this filter with respect to molecular size.
    \item From \hyperref[fig:fig-3]{Fig.3}, it can be understood that 49.08\% of the dataset passes 3 out of 4 filters, followed by 29.60\% passing all 4 filters and only 4.82\% failing all four. This distribution indicates that the majority of FDA-approved drugs comply with most, but not necessarily all, established drug-likeness rules.
\end{enumerate}

\section{Conclusion}
The overall study reflects agreement between the Lipinski, Veber and Egan filters and a significant disagreement and deviation for the Ghose filter. Further analysis shows the chances of failing all four filters is merely 4.82\%, whereas the chances of passing at least 3 filters are 48.79\%. Instances of failures can also stem from drugs targeting the central nervous system receptors, having significantly different properties than conventional oral drugs. This implies two major points -
\begin{itemize}
    \item Most FDA-approved drugs comply with the 3 or more filters, thus emphasising the viability of computationally analysed drug-likeness.
    \item However, drugs tend to fail certain filters but pass certain others, making them filter-dependent, rather than absolute.
\end{itemize}
The study reinforces that these physicochemical guidelines are only an early screening tool and do not provide definite predictions for clinically successful drugs. Rigid adherence to these results can result in rejection of potentially effective drug candidates.



\end{document}
